\section{Example Usage}\label{example}

Suppose you have a \C function \code{int fib(int n)} which produces the n'th 
Fibonacci number:
\begin{lstlisting}[language=fanC,title=Fibonacci]
int fib(n)
{
  if (n == 0 || n == 1)
    return 1;
  else
    return fib(n-1) + fib(n-1);
}
\end{lstlisting}

To wrap this for \R with \thispackage, you 
\begin{lstlisting}[language=fanC,title=Fibonacci Wrapper]
#include <SEXPtools.h>

SEXP fib_wrap(SEXP n)
{
  R_INIT;
  SEXP ret;
  newRvec(ret, 1, "int");
  
  INT(ret) = fib(INT(n));
  
  R_END;
  return ret;
}
\end{lstlisting}

This may look verbose --- and really it is --- but for complicated examples, 
the tools begin to shine.  Suppose we wanted, for example, to return both n and 
the n'th Fibonacci number in a list.  In \R's native \C interface, this is a 
labyrinthine nightmare, but managing return lists is very simple with 
\thispackage:

\begin{lstlisting}[language=fanC,title=Fibonacci Wrapper 2]
#include <SEXPtools.h>

SEXP fib_wrap(SEXP n)
{
  R_INIT;
  SEXP R_list, R_list_names, nth_fib;
  newRvec(ret, 1, "int");
  
  INT(nth_fib) = fib(INT(n));
  
  R_list_names = make_list_names(2, "n", "nth.fib");
  R_list = make_list(R_list_names, 2, n, nth_fib);
  
  R_END;
  return R_list;
}
\end{lstlisting}

You probably know that really I should be instituting a copy of n here; this is 
true of \pkg{Rcpp} and the pure native \C interface as well.

Note that \code{INT(n)} is a shorthand for \code{INT(n, 0)}, which is itself 
shorthand for \R's \code{INTEGER(n)[0]}.  Other values may be substituted, e.g. 
\code{INT(n, i)} for \code{INTEGER(n)[i]}.  See \secref{specification} for more 
information.