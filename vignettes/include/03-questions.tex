\section{Q\&A}

This section is a set of frequently asked questions (FAQ), with frequency uniformly
equal to zero.


\subsection{Why make this?}

Probably my biggest motivator was fun; I just wanted to make something like this.  Another,
more pragmatic reason is that part of my workload (for very non-standard reasons not worth 
getting into) prevents me from using \pkg{Rcpp}.  This leaves me stuck with the native \C\ 
interface for \R.  And I don't like \R's native \C\ interface.  This is my attempt to 
make that interface (slightly) more friendly.


\subsection{Why the strange name?}

Every \R\ object (underneath, in the \C\ interface) is an \code{SEXP} (short for
S-expression) object, which is a struct pointer.  This is explained in the
\href{http://cran.r-project.org/doc/manuals/R-ints.html#SEXPs}{R Internals} 
manual.


\subsection{How does this differ from \pkg{Rcpp}?}
Each of these packages makes an attempt at solving a serious problem with utilizing compiled
code from \R: the native interface for \proglang{C} code in \R\ sucks.  There are huge 
differences between the two packages, however. In short, \pkg{Rcpp} is \emph{much} a much 
more comprehensive solution.  If you are new to using compiled code with \R, frankly this 
package probably is not for you; you would likely be much better served by \pkg{Rcpp}.  
However, if for some combination of reasons you either cannot or prefer to not use 
\pkg{Rcpp}, then this package may be of interest to you.

Beyond the scope and ease of use of each project (where \pkg{Rcpp} handily wins), there are
some other critical differences between the projects.  A few of note are:

\begin{enumerate}
  \item \thispackage\ is more permissively licensed than \pkg{Rcpp} (BSD rather than GPL)
  \item \thispackage\ is pure \C\, while \pkg{Rcpp} is \Cpp.
\end{enumerate}

These things may not matter in the least to you.  If that's the case, then you
may well be better served by \pkg{Rcpp}.



\subsection{Why would I want to use it?}

You may well not.  But it is an option available to you.


\subsection{How would I use \thispackage\ in a package?}

