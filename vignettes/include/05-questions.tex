\section{Q\&A}

\subsection{Why make this?}
Probably my biggest motivator was fun; I wanted to.  
Another, more pragmatic reason is that part of my workload prevents me from 
using \pkg{Rcpp}.  I deal a lot with \C and \Fortran; trying to integrate \Cpp 
into that mess is not fun, and so for me, \pkg{Rcpp} is more of a burden than a 
savior.  This leaves me stuck with the native \C interface for \R.  And I don't 
like \R's native \C interface.  This package is my attempt to make that 
interface (slightly) more friendly and convenient to work with.


\subsection{Why the strange name?}
Every \R object (underneath, in the \C interface) is an \code{SEXP} (short for
S-expression) object, which is a struct pointer.  This is explained in the
\href{http://cran.r-project.org/doc/manuals/R-ints.html#SEXPs}{R Internals} 
manual.  This package is a collection of tools for more easily managing SEXP 
objects.


\subsection{Is this now, or will this ever be a competitor to \pkg{Rcpp}?}
In terms of features, no.  \pkg{Rcpp} is very good at aiding the writing of
new code, or translating \R code into compiled code.  That is not a goal
of this package.


\subsection{How does this differ from \pkg{Rcpp}?}\label{sec:rcppdiffs}
Each of these packages makes an attempt at solving a serious problem with 
utilizing compiled code from \R: the native interface for \proglang{C} code in 
\R sucks.  There are huge differences between the two packages, however. In 
short, \pkg{Rcpp} is \emph{much} a much more comprehensive solution.  If you are 
new to using compiled code with \R, frankly this package probably is not for 
you; you would likely be much better served by \pkg{Rcpp}.  

\thispackage is primarily meant for those who are working in pure \C, writing 
\C/\Fortran code and wrapping it back into \R.  This makes the wrapping a bit 
simpler.  It does not make the translation of \R code into compiled code any 
simpler, which is one of \pkg{Rcpp}'s main goals.

\thispackage is a pure \C solution.  \pkg{Rcpp} is a \Cpp solution.  In the author's 
opinion, \Cpp brings a lot of needless complications and headaches if you aren't
actually using \Cpp; that is, if you are only writing \C and \Fortran, then 
bringing in \Cpp \textbf{will} create problems.  

Another important difference is that \thispackage is more permissively 
licensed than \pkg{Rcpp} (BSD rather than GPL).  If this issue is important to 
you but you live in the \Cpp world, you may be interested in Romain Francois' 
new \pkg{Rcpp11} package.



\subsection{Why would I want to use this package?}
If spend a lot of time bringing code to \R (especially if you deal with lists 
and dataframes), this package offers a lot of convenient shorthand for doing so.



\subsection{Is this really easier than R's native interface?}
It is for me, no question; notably, returning lists and dataframes is 
\emph{much} less painful through some
\href{https://en.wikipedia.org/wiki/Stdarg.h}{\code{stdarg}}
sorcery.  Most of 
the rest of the package is minor cosmetic things; but the package is very 
permissively licensed, so feel free to pick and choose what you want, however 
you want.


\subsection{How would I use \thispackage\ in a package?}
Assuming that you have some compiled code you have or want to create to use with 
a package, you simply link with \thispackage\ and then wrap that compiled code 
with the utilities provided by \thispackage.  For the former, see 
\secref{linking}, and for the latter, see \secref{specification} and 
\secref{example}.  For actual package examples using \thispackage, see 
\href{http://cran.r-project.org/web/packages/pbdBASE/index.html}{\pkg{pbdBASE}} 
version~$\geq$~0.3-0,  
\href{http://cran.r-project.org/web/packages/pbdDMAT/index.html}{\pkg{pbdDMAT}}
version~$\geq$~0.3-0, and
\href{http://cran.r-project.org/web/packages/memuse/index.html}{\pkg{memuse}}
version~$\geq$~2.0.

Philosophically, you should never have the bulk of the work of a function (of any
importance) be handled by the \R interface (including \thispackage's version of 
it). If you do, then your code can never (easily) have a life outside of \R.  
That may sound fine to you now, but if you ever decide to take some of your 
work outside of \R, then you can't easily take your compiled code with you.  
This is just bad practice and shortsightedness.  
